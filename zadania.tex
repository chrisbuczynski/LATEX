%\documentclass[a4paper,12pt]{article}
%\usepackage{amssymb}
%\usepackage{amsmath}
%\usepackage{amsthm}
%\usepackage[T1]{fontenc}		%% wgrywa specjalne znaki polskie
%\usepackage[utf8]{inputenc}		%% pozwala na dobre kodowanie (dla Linuxa: utf8)
%\usepackage[polish]{babel}  	%% warto mieć do pakietu polskiego
%\usepackage{polski}			%% wgrywa łamanie w języku polskim


%\usepackage{pstricks, pst-plot}
%\usepackage{graphicx}
%\usepackage[top=2cm, bottom=2cm, left=2cm, right=2cm]{geometry}	

%\usepackage{enumerate}	%żeby można było dobrze numerowane i zmienia style num.
%\usepackage{fancyhdr}	%żeby można było ustawiać swoje nagłówki stopki
%\usepackage{amsfonts}	%żeby działały polecenia \mathbb w def. np. NN
%\usepackage{hyperref}	%żeby działały linki www

%%%% POZWALA KOMPILOWAÆ NAWET JEŒLI S¥ DROBNE B£ÊDY %%%% 
%\nonstopmode

%\newcommand{\NN}{\mathbb{N}}
%\newcommand{\CC}{\mathbb{C}}
%\newcommand{\ZZ}{\mathbb{Z}}
%\newcommand{\NW}{\mathbb{NW}}
%\newcommand{\WW}{\mathbb{W}}
%\newcommand{\QQ}{\mathbb{Q}}
%\newcommand{\IQ}{\mathbb{IQ}}
%\newcommand{\RR}{\mathbb{R}}

%\newcommand{\al}{\alpha}
%\newcommand{\be}{\beta}
%\newcommand{\ga}{\gamma}

%\newtheorem{zada}{Zadanie}
%\newtheorem{rozw}{Rozwiązanie}
%\newtheorem{uw}{Uwaga}

%\usepackage{parskip} %% dodaje odstęp w paragrafach

%\begin{document}

\begin{zada}
Rozwiąż nierówność:
\[\frac{4x^2}{(1-\sqrt{1+2x})^2} < 2x+9 \]
\end{zada}

$ $

\begin{zada} 
Dla jakich wartości parametru $m$ funkcja $f(x)=\frac{x^2+(m+1)x+m}{x^2-9}$ ma dokładnie jedno miejsce zerowe?
\end{zada}


$ $

\begin{zada}\footnote{Zadanie jest wariacją na teamat anegdotki o Eulerze. Więcej anegdot można znaleźć na stronie:  https://www.tomaszgrebski.pl/blog/matematyka-na-wesolo/anegdoty-matematyczne}
Dwa osły oddalone od siebie o~100 metrów wyruszają w~tym samym momencie i~idą wprost na~przeciwko siebie ze stałą prędkością $1~m/s.$ Mucha siedząca na nosie pierwszego osła startuje równo z~osłami i~leci do nosa drugiego, po czym od razu zawraca do nosa pierwszego osła, po czym wraca do drugiego itd. Prędkość muchy to $10~m/s.$ Pytanie brzmi: ile czasu minie zanim mucha zostanie zmiażdżona między dwoma nosami osłów?\\ 
\end{zada}

$ $

\begin{zada}
Rozwiąż układ równań:

\[ \begin{cases}
x \cdot y &= z + 2y  \\
\sqrt{z} &= 2y \\
x \cdot z - 1984 &= 2z + 2 \cdot \frac{80}{\sqrt{z}} 
\end{cases} \]

\end{zada}

$ $

\begin{zada} \footnote{Zadanie pochodzi z książki fabularnej, niestety nie znam tytułu ani autora. Zostało mi kiedyś przesłane w ramach ciekawostki. Gdyby ktoś z czytelników znał źródło tego zadania, bardzo proszę o kontakt.}
Współczynniki $a$~i~$b$ są liczbami rzeczywistymi i~spełniają warunek: $0<a<b.$ Zakładamy, że $u_{0}=a$ i~$v_{0}=b$ dla całego ciągu liczb naturalnych $n$:

\[ u_{n+1}=\frac{u_{n}+v_{n}}{2} \ \textrm{ oraz } \ v_{n+1}=\sqrt{u_{n+1}v_{n}}\]

Udowonij, że ciągi $u_{n}$ oraz $v_{n}$ są zbieżne i~że ich wspólna granica jest równa

\[ \frac{b\sin(\arccos(\frac{a}{b}))}{\arccos(\frac{a}{b})} \]
\end{zada}

$ $

\begin{zada} %Zadanie od Zbyszka fizyka ze studiów podyplomowych
Rozwiąż nierówność
\[ \Big|\frac{2x-3}{x^2-1}\Big| \geq 2 \]
\end{zada}

$ $

\begin{zada}
Rozwiąż nierówność $|\frac{1}{x-2}|>|\frac{1}{x+1}|$
\end{zada}

$ $

\begin{zada}
Rozwiąż równanie $4^x + 6^x = 9^x$
\end{zada}

$ $

\begin{zada}
Rozwiąż równanie $\dfrac{8^x+27^x}{12^x+18^x}=\dfrac{7}{6}$
\end{zada}

$ $

\begin{zada}
Rozwiąż równanie $1+\sin6x = \sin3x + \cos3x$ w~przedziale $[0,2\pi]$
\end{zada}

$ $

\begin{zada}
Wiedząc, że 
\[\frac{1}{\sqrt[3]{25}+\sqrt[3]{5}+1} = A\sqrt[3]{25}+B\sqrt[3]{5}+C, \]
gdzie $A,B,C$ są liczbami wymiernymi, znajdź wartość wyrażenia $A+B+C$. 
\end{zada}

$ $

\begin{zada}
Znajdź dokładną wartość wyrażenia 
\[ \sum_{k=-9}^{9} \frac{1}{10^{k}+1}\]
\end{zada}

    
%\end{document}
